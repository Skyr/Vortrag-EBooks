\begin{frame}{Der ePub-Standard}
	\begin{itemize}
		\item Flexible Darstellung von Büchern und sonstigen Publikationen auf verschiedenen Geräten
		\item Reflow: Layout-Umbruch passend für die Display-Größe
		\item Standard besteht aus drei Teilen:
		\begin{itemize}
			\item OCF -- Open Container Format
			\item OPF -- Open Packaging Format
			\item OPS -- Open Publication Structure
		\end{itemize}
	\end{itemize}
\end{frame}

\begin{frame}{Open Container Format}
Verzeichnisstruktur:
\dirtree{%
.0 EPub-Datei\DTcomment{(Zip-Archiv)}.
.1 mimetype.
.1 META-INF.
.2 container.xml.
.2 (weitere optionale Dateien)\DTcomment{(z.B. für DRM)}.
.1 OEBPS.
.2 content.opf.
.2 toc.ncx.
.2 (Daten).
}
OCF definiert Archivtyp, Regeln für Dateinamen, Inhalt von container.xml
\end{frame}


\begin{frame}[fragile]
\frametitle{Open Container Format}
container.xml:
\begin{xml}
<?xml version="1.0"?>
<container version="1.0" xmlns="urn:oasis:names:tc:opendocument:xmlns:container">
    <rootfiles>
        <rootfile full-path="OEBPS/content.opf" media-type="application/oebps-package+xml"/>
   </rootfiles>
</container>
\end{xml}
	\begin{itemize}
		\item Beschreibt Inhalt des Containers
		\item Definiert Version des Containerformats
		\item Beinhaltet typischerweise genau eine rootfile-Referenz
		\item Mehrere rootfile-Referenzen: Alternative Formate (z.B. PDF)
	\end{itemize}
\end{frame}

\begin{frame}{Open Packaging Format}
	\begin{itemize}
		\item OPF Package Document (.opf-Datei)
		\begin{itemize}
			\item Eindeutige Package-ID
			\item Metadaten des E-Books
			\item Manifest: Liste aller verwendeten Ressourcen
			\item Spine: Dokumente in linearer Lesereihenfolge
			\item Guide: Referenzen auf besondere Abschnitte wie Vorwort, Literaturangaben, etc.
		\end{itemize}
		\item Navigation Center eXtended (.ncx-Datei)
		\begin{itemize}
			\item Hierarchische Gliederung des Inhalts
			\item Früher: Guide-Abschnitt in .opf-Datei
		\end{itemize}
	\end{itemize}
\end{frame}

\begin{frame}[fragile]
\frametitle{Open Packaging Format}
content.opf:
\begin{xml}
<?xml version="1.0" encoding="UTF-8"?>
<package xmlns="http://www.idpf.org/2007/opf" unique-identifier="BookID" version="2.0">
  <metadata xmlns:dc="http://purl.org/dc/elements/1.1/" xmlns:opf="http://www.idpf.org/2007/opf">
      <dc:title>Testbuch</dc:title>
      <dc:identifier id="BookID" opf:scheme="UUID">urn:uuid:6eeee(...)</dc:identifier>
      <!-- Weitere Infos im Dublin Core Format -->
  </metadata>
  <manifest>
    <item id="ncx" href="toc.ncx" media-type="application/x-dtbncx+xml"/>
    <item id="bild.j2k" href="Images/bild.j2k" media-type="image/jpeg2000" fallback="bild.jpg"/>
    <item id="bild.jpg" href="Images/bild.jpg" media-type="image/jpeg"/>
    <item id="Section0001.xhtml" href="Text/Section0001.xhtml" media-type="application/xhtml+xml"/>
  </manifest>
\end{xml}
	\begin{itemize}
		\item Fallback: Dateien in einem Nicht-Standard-Format
	\end{itemize}
\end{frame}

\begin{frame}[fragile]
\frametitle{Open Packaging Format}
content.opf:
\begin{xml}
  <spine toc="ncx">
    <itemref idref="Section0001.xhtml"/>
  </spine>
	<guide>
		<reference type="index" title="Stichwortverzeichnis" href="bookindex.html"/>
	</guide>
</package>
\end{xml}
	\begin{itemize}
		\item toc-Attribut des Spine: ID, welche auf ein NCX-Inhaltsverzeichnis verweist
		\item Guide-Typen: title-page, toc, index, glossary, uvm.
	\end{itemize}
\end{frame}

\begin{frame}[fragile]
\frametitle{Open Packaging Format}
toc.ncx:
\begin{xml}
<?xml version="1.0" encoding="UTF-8"?>
<!DOCTYPE ncx PUBLIC "-//NISO//DTD ncx 2005-1//EN" "http://www.daisy.org/z3986/2005/ncx-2005-1.dtd">
<ncx xmlns="http://www.daisy.org/z3986/2005/ncx/" version="2005-1">
  <head>
    <meta name="dtb:depth" content="2"/>
    <meta name="dtb:totalPageCount" content="0"/>
    <meta name="dtb:maxPageNumber" content="0"/>
  </head>
  <docTitle>
    <text>Testbuch</text>
  </docTitle>
  <navMap>
    <navPoint id="navPoint-1" playOrder="1">
      <navLabel>
        <text>Titel 1</text>
      </navLabel>
      <content src="Text/Section0001.xhtml"/>
      <navPoint id="navPoint-2" playOrder="2">
        <navLabel>
          <text>Abschnitt 1</text>
        </navLabel>
        <content src="Text/Section0001.xhtml#chapter"/>
      </navPoint>
    </navPoint>
  </navMap>
</ncx>
\end{xml}
\end{frame}

\begin{frame}{Open Publication Structure}
	\begin{itemize}
		\item Definiert XHTML/CSS-Subset und Grafikformate
		\item XHTML: Diverse Tags "`bevorzugt"'
		\item CSS
		\begin{itemize}
			\item Standard listet div. unterstützte Attribute
			\item Einige zusätzliche Attribute für ePub:
			\item oeb-page-head, oeb-page-foot: Elemente (z.B. div), die als Kopf/Fuß gezeigt werden
			\item oeb-column-number: Erlaubt mehrspaltiges Rendering
		\end{itemize}
	\end{itemize}
\end{frame}

% vim: ai ts=2 sw=2
