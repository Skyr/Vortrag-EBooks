\begin{frame}{Selbstleuchtende Displays}
	\begin{columns}
		\column{.7\textwidth}
			Stark vereinfachter Aufbau:
			\begin{itemize}
				\item Hintergrundbeleuchtung
				\item Polarisationsfilter
				\item Flüssigkristallschicht
				\item Farbfilter
			\end{itemize}
			Beim iPad: "`Retina-Display"', IPS-Technik, Auflösung $>$300 dpi
		\column{.3\textwidth}
			\pgfimage[width=\textwidth]{bilder/LcdDeadPixel}
	\end{columns}
\end{frame}

\begin{frame}{Reflektive Displays}
	\begin{itemize}
		\item Mit LCD lassen sich auch reflektive (oder transreflektive) Displays bauen
		\item Wegen höherem Kontrast kommen e-Paper-Technologien zum Einsatz
		\item eInk ist übrigens eingetragenes Warenzeichen!
		\item Erste e-Papers: Drehbare Kügelchen (Gyricon)
		\item Heute bei fast allen e-Papers: Elektrophorese
	\end{itemize}
\end{frame}

\begin{frame}{Reflektive Displays}
	\begin{columns}
		\column{.6\textwidth}
			\begin{itemize}
				\item Reflektierender Hintergrund (8)
				\item Pixel-Elektroden-Schicht (7)
				\item Mikrokapseln (3)
				\item Schwarze und weiße Pigmente (4, 5), transparentes Öl (6)
				\item Transparente Elektrodenschicht (2)
				\item Oberflächenschutz, ggf. Farbfilter (1)
			\end{itemize}
		\column{.5\textwidth}
			\pgfimage[width=\textwidth]{bilder/Electronic_paper_Sideview}
	\end{columns}
\end{frame}

\begin{frame}{E-Ink (Kindle) unter dem Mikroskop}
	\begin{center}
		\pgfimage[width=.9\textwidth]{bilder/flickr-28594931@N03-5038428724}
	\end{center}
\end{frame}

\begin{frame}{Displaytechnologien im Vergleich}
	\begin{columns}
		\column{.45\textwidth}
			Selbstleuchtende Displays

			\begin{itemize}
				\item Schnelle Schaltzeiten (Animationen möglich)
				\item Farbe, hohe Farbbrillianz
				\item Hohe Auflösung
				\item Schlecht lesbar in heller Umgebung
				\item Ermüdet beim Lesen in dunkler Umgebung
			\end{itemize}

			\vfill
		\column{.55\textwidth}
			E-Paper-Displays

			\begin{itemize}
				\item Langsame Schaltzeiten (neue Controller verkürzen diese)
				\item Stromfluß nur bei Bildaufbau
				\item Leseeigenschaften wie Papier
				\item Geringere Auflösung (ca.\,200~dpi)
				\item Farbdisplays erst kurz vor Serienreife
			\end{itemize}

			\vfill
	\end{columns}
\end{frame}

% vim: ai ts=2 sw=2
