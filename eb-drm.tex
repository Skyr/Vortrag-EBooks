\begin{frame}{DRM -- So funktioniert's (oft)}
	\begin{center}
		\def\svgwidth{10cm}
		\input{drmschema1.pdf_tex}
	\end{center}
\end{frame}

\begin{frame}{DRM mit Adobe Digital Edition (vermutlich)}
	\begin{center}
		\def\svgwidth{10cm}
		\input{drmschema2.pdf_tex}
	\end{center}
\end{frame}

\begin{frame}{Onleihe: DRM mit Adobe Digital Edition (vermutlich)}
	\begin{center}
		\def\svgwidth{10cm}
		\input{drmschema3.pdf_tex}
	\end{center}
\end{frame}

\begin{frame}{DRM und E-Books}
	\begin{itemize}
		\item Notwehr der Publizisten?
		\begin{itemize}
			\item Bindung von Geräten an einen Nutzer
			\item Lizensierung von Büchern an einen Nutzer
			\item Eigene Geräte können eigene Bücher ansehen
		\end{itemize}
		\item<2-> ...oder "`feuchter Traum"' der Publizisten?
		\begin{itemize}
			\item<2-> Geräte prüfen regelmäßig beim DRM-Server Legitimität des Zugriffs auf ein Buch
			\item<2-> Verleih von Büchern zwischen Nutzern
			\item<2-> Andere Geschäftsmodelle, z.B. Buchverleih
			\item<2-> "`Bestrafung"' von Lizenzübertretungen durch Deaktivieren des Zugriffs
			\item<2-> "`Rückruf"' von Inhalten (aus lizenzrechtlichen, politischen oder sonstigen Gründen)
		\end{itemize}
	\end{itemize}
\end{frame}

\begin{frame}{DRM und E-Books}
	\begin{block}{ }
		"`You cannot own an eBook. You can only license it."' \\
		--- Brit. Verleger
	\end{block}
\end{frame}


\newcommand{\cellyes}{\cellcolor{red}Ja}
\newcommand{\cellunknown}{\cellcolor{yellow}?}
\newcommand{\cellno}{\cellcolor{green}Nein}

\begin{frame}{Verschiedene DRM-Mechanismen}
	\begin{tabular}{lcccc}
	Verfolgbar/Trackbar & Kindle & Sony & iPad & Adobe \\
	Büchersuche & \cellunknown & \cellunknown & \cellyes & \cellno \\
	Buchkäufe & \cellyes & \cellno & \cellyes & \cellno \\
	Gelesenes Buch & \cellyes & \cellno & \cellno & \cellno \\
	Infos an Dritte & \cellyes & \cellyes & \cellyes & \cellno \\
	\end{tabular}
\end{frame}

\begin{frame}{Und was, wenn...}
	\begin{itemize}
		\item ...ich mein System neu aufsetzen muß? Backups! Geräteaktivierung!
		\item ...der DRM-Server nicht erreichbar ist oder abgeschaltet wurde?
		\item ...ich mein E-Book ausleihen möchte?
		\item ...ich mein E-Book verkaufen will?
		\item ...ich Bücher zu "`brisanten"' Themen lese?
		\item ...aufgrund von Repressionen ein Buch verschwinden soll?
	\end{itemize}
\end{frame}

\begin{frame}{DRM und E-Books}
	\begin{block}{Doctorow's First Law}
		"`Any time someone puts a lock on something that belongs to you, and won’t give you a key, they’re not doing it for your benefit."'
	\end{block}
	\begin{block}{"`The danger of e-books"', R. Stallman}
		"`We must reject ebooks until they respect our freedom. (...) 
		Ebooks need not attack our freedom, but they will if companies get to decide. It's up to us to stop them. The fight has already started."'
	\end{block}
\end{frame}

% vim: ai ts=2 sw=2
