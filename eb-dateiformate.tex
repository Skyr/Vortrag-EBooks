\begin{frame}{Futter für den E-Book-Reader: Dateiformate}
	Die verbreitetsten Formate:

	\begin{itemize}
		\item PDF (Portable Document Format)
		\item azw / topaz (Kindle-eigene Dateiformate)
		\item mobi (Mobipocket -- Basis für die Kindles azw)
		\item ePub (Electronic Publication)
	\end{itemize}

	...und viele weitere (Plucker, etc.).

	Sonderrolle: SiSu (Structured Information, Serialized Units): Meta-Format zum Generieren verschiedener Ausgabeformate
\end{frame}

\begin{frame}{Das "`ich-kann-alles"'-Format: PDF}
	\begin{itemize}
		\item Format seit 1993 von Adobe entwickelt
		\item Anfangs Nutzung gegen Lizenz, seit 2008 ISO-Standard
		\item Ursprünglicher Zweck: Geräteunabhängige Dokumentendarstellung, 100\%ige Layout-Treue
		\item Ständig wachsender Feature-Satz:
		\begin{itemize}
			\item Editierbarkeit, Kommentare
			\item Formulare
			\item Interaktivität mittels Javaskript
			\item Multimedia-Inhalte, 3d-CAD-Daten, Flash
			\item Reflow (Aufbrechen des Layouts)
			\item ...und vieles mehr
		\end{itemize}
	\end{itemize}
\end{frame}

\begin{frame}{ePub -- die "`Electronic PUBlication"'}
	\begin{itemize}
		\item Erste Spezifikation von 1999. v2.0 von 2007, aktuell: v2.0.1
		\item Relativ simples Format, auf den Zweck zugeschnitten
		\item Wiederverwendung vieler Standards und -technologien
		\begin{itemize}
			\item Konsequente Verwendung von Unicode
			\item Containerformat: zip
			\item Beschreibungsdaten: XML
			\item Inhalte: XHTML 1.1, CSS-2.0-Subset
			\item Grafikformate: png, jpeg, gif, svg
		\end{itemize}
	\end{itemize}
\end{frame}

% vim: ai ts=2 sw=2


% vim: ai ts=2 sw=2
