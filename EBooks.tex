\documentclass[hyperref={pdfpagelabels=false}]{beamer}
%\documentclass{beamer}
%\documentclass[draft]{beamer}

\usepackage[german]{babel}
\usepackage[utf8]{inputenc}
\usepackage{url}
\usepackage{endnotes}
\usepackage{calc}

\usetheme{Antibes}
\usecolortheme{dolphin}
\usefonttheme{default}
\setbeamertemplate{sidebar left}{\vfill\hspace{0.1cm}\pgfimage[height=.5cm]{bilder/CC-BY-NC-SA}}
\setbeamertemplate{footline}[frame number]
\setbeamercovered{transparent}

% Fixing Warnings
\RequirePackage{fix-cm}
\RequirePackage{pgfcore}
\usepackage{pgfpages} 
\let\Tiny=\tiny
\usepackage{lmodern}

%\newcommand{\email}[2]{#1@#2}
\newcommand{\email}[2]{#1\pgfimage[height=2ex]{bilder/at_sign}#2}
\newcommand{\urlref}[2]{\only<#1>{\endnote{\url{#2}}}}
%\def\supertiny{ \font\supertinyfont = cmr10 at 1pt \relax \supertinyfont} 
\renewcommand{\enotesize}{\tiny}

\title{E-Books}
\subtitle{Einblick in die Technik der digitalen Bücher}
\author{%
		Stefan Schlott%
}
\institute{%
	Web: \url{http://stefan.ploing.de/} \\
	Twitter: \href{https://twitter.com/\_skyr}{@\_skyr}
}
\date{12.05.2011}
\logo{\vspace{5.5cm} \pgfimage[height=2cm]{bilder/Chaosknoten}}


\begin{document}

\begin{frame}[plain]
\titlepage
\end{frame}

\begin{frame}{Worum es gehen wird...}
	\begin{itemize}
		\item Rise of the E-Book
		\item Ein wenig Technik - Displays und Dateiformate
		\item ePub - das wohl populärste Format
		\item DRM mit Adobe Adept
		\item Etwas Science Fiction
	\end{itemize}
\end{frame}


\section{Geschichte}

\begin{frame}{Rise of the E-Book}
	\begin{itemize}
		\item 1968: Alan Kay: Dynabook% Technische Dokumentation Panzer
		\item 1971(!): Gründung "`Project Gutenberg"'% (einer der ersten 15 Rechner des Internets)
		\item 1993: Erster Verkauf von E-Books
		\item 1998: Erste ISBN für ein E-Book vergeben
		\item<2-> 1998: "`Rocket eBook"' -- 5,5\dq LCD-Screen, 620\,g, 499\,\$
		\item<2-> 2000: Mobipocket vertreibt Readersoftware (Palm, Windows CE, etc.)
		\item<2-> 2002: Randomhouse und HarperCollins verkaufen digitale Versionen ihrer Bücher
		\item<3-> 2005: Amazon kauft Mobipocket
		\item<3-> 2006: Sony PRS-500 -- 6\dq eInk-Display, 250\,g, 350\,\$
		\item<3-> ab 2007: iPhone und Android -- Readersoftware
		\item<3-> 2007: Amazon Kindle
		\item<4-> 2010: Apple iPad
		\item<4-> Q2/2010: Amazon verkauft mehr E-Books als Hardcovers
	\end{itemize}
\end{frame}

\begin{frame}{Historische Geräte}
	\begin{columns}
		\column{.5\textwidth}
			\pgfimage[width=\textwidth]{bilder/Dynabook} \\
			Skizze Dynabook
		\column{.5\textwidth}
			\pgfimage[width=\textwidth]{bilder/flickr-81374383@N00-2262899712} \\
			Sony Librié (?), Rocketbook
	\end{columns}
\end{frame}

\begin{frame}{Aktuelle Reader}
	\begin{columns}
		\column{.5\textwidth}
			\pgfimage[height=.6\textheight]{bilder/flickr-40732538582@N01-3624716229} \\
			Sony PRS-Serie
		\column{.5\textwidth}
			\pgfimage[height=.6\textheight]{bilder/flickr-94168846@N00-4797684465} \\
			Kindle
	\end{columns}
\end{frame}

\begin{frame}{Der "`Heilbringer"'}
	\begin{columns}
		\column{.4\textwidth}
			\pgfimage[width=\textwidth]{bilder/flickr-16226024@N00-4508917013} \\
			Apple iPad
		\column{.6\textwidth}
			\begin{itemize}
				\item iPad wird bei der Ankündigung (auch) als \href{http://www.youtube.com/watch?v=gew68Qj5kxw}{E-Book-Lesegerät entdeckt}
				\item {\em "`Every publisher in the world should sit down once a day and pray to thank Steve Jobs that he is saving the publishing industry"'} -- M. Döpfner, CEO Axel Springer
				\item Durch Apples Medienpräsenz: Spätestens jetzt sind E-Books mainstream
			\end{itemize}
	\end{columns}
\end{frame}

\begin{frame}{Apples Politik}
	\begin{columns}
		\column{.3\textwidth}
			\pgfimage[width=\textwidth]{bilder/Steve_Jobs_with_the_Apple_iPad_no_logo} \\
		\column{.7\textwidth}
			\begin{itemize}
				\item E-Book-Reader mit Shopanbindung erscheinen fürs iPad. Bezahlung über Webseite der Betreiber
				\item Apple ändert AGBs: Extern zukaufbare Inhalte müssen zum selben Preis via App-Store verfügbar sein
				\item App-Store-Kauf: 30\% des Betrags geht an Apple
				\item Das ist bei Buchhändlern mehr als die Gewinnmarge
				\item Apple startet eigenen Bookstore
			\end{itemize}
	\end{columns}
\end{frame}

\section{Technik}


\section{ePub}


\section{DRM}


\section{Ausblick}


\begin{frame}[plain]
	That's it!
	\begin{center}
		\pgfimage[height=6cm]{bilder/chaosknoten-fernsehturm}
	\end{center}
\end{frame}

\begin{frame}{Image credits}
  \tiny
  \begin{itemize}
    \item \url{https://secure.wikimedia.org/wikipedia/en/wiki/File:Dynabook.png} Wikipedia Fair use
    \item \url{http://www.flickr.com/photos/cloudsoup/3624716229/} David Jones (CC) by
    \item \url{http://www.flickr.com/photos/librarycommission/2262899712/} Nebraska Library Commission (CC) by-nc-sa
    \item \url{http://www.flickr.com/photos/jimmiehomeschoolmom/4797684465/} Jimmie (CC) by
  \end{itemize}
\end{frame}


\end{document}
% vim: ai ts=2 sw=2

