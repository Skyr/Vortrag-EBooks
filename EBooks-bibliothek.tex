\documentclass[hyperref={pdfpagelabels=false}]{beamer}
\providecommand\thispdfpagelabel[1]{}
%\documentclass{beamer}
%\documentclass[draft]{beamer}

\usepackage[german]{babel}
\usepackage[utf8]{inputenc}
\usepackage{url}
\usepackage{endnotes}
\usepackage{calc}
\usepackage{listings}
\usepackage{dirtree}
\usepackage{color}
\usepackage{colortbl}

\usetheme{Antibes}
\usecolortheme{dolphin}
\usefonttheme{default}
\setbeamertemplate{sidebar left}{\vfill\hspace{0.1cm}\pgfimage[height=.5cm]{bilder/CC-BY-NC-SA}}
\setbeamertemplate{footline}[frame number]
\setbeamercovered{transparent}

% Fixing Warnings
\RequirePackage{fix-cm}
\RequirePackage{pgfcore}
\usepackage{pgfpages} 
\let\Tiny=\tiny
\usepackage{lmodern}

%\newcommand{\email}[2]{#1@#2}
\newcommand{\email}[2]{#1\pgfimage[height=2ex]{bilder/at_sign}#2}
\newcommand{\urlref}[2]{\only<#1>{\endnote{\url{#2}}}}
%\def\supertiny{ \font\supertinyfont = cmr10 at 1pt \relax \supertinyfont} 
\renewcommand{\enotesize}{\tiny}

\lstnewenvironment{xml}
  {\lstset{language=XML,
		basicstyle=\tiny,
		extendedchars=\true,
		tabsize=2,
		breaklines=true,
		breakatwhitespace=true,
		inputencoding=utf8}}
	{}


\title{E-Books}
\subtitle{Einblick in die Technik der digitalen Bücher}
\author{%
		Stefan Schlott%
}
\institute{%
	Web: \url{http://stefan.ploing.de/} \\
	Twitter: \href{https://twitter.com/\_skyr}{@\_skyr}
}
\date{03/2012}
\logo{\vspace{5.5cm} \pgfimage[height=2cm]{bilder/Chaosknoten}}


\begin{document}

\begin{frame}[plain]
\titlepage
\end{frame}


\begin{frame}{Worum es gehen wird...}
	\begin{itemize}
		\item Geschichte des E-Books
		\item Ein wenig Technik - Displays und Dateiformate
		\item DRM mit Adobe Adept
		\item Versuch eines Blicks in die Glaskugel
	\end{itemize}
\end{frame}


\section{Geschichte}

\begin{frame}{Rise of the E-Book}
	\begin{itemize}
		\item 1968: Alan Kay: Dynabook% Technische Dokumentation Panzer
		\item 1971(!): Gründung "`Project Gutenberg"'% (einer der ersten 15 Rechner des Internets)
		\item 1993: Erster Verkauf von E-Books
		\item 1998: Erste ISBN für ein E-Book vergeben
		\item<2-> 1998: "`Rocket eBook"' -- 5,5\dq LCD-Screen, 620\,g, 499\,\$
		\item<2-> 2000: Mobipocket vertreibt Readersoftware% (Palm, Windows CE, etc.)
		\item<2-> 2002: Randomhouse und HarperCollins verkaufen digitale Versionen ihrer Bücher
		\item<3-> 2005: Amazon kauft Mobipocket
		\item<3-> 2006: Sony PRS-500 -- 6\dq eInk-Display, 250\,g, 350\,\$
		\item<3-> ab 2007: iPhone und Android -- Readersoftware
		\item<3-> 2007: Amazon Kindle
		\item<4-> 2010: Apple iPad
		\item<4-> Q2/2010: Amazon verkauft mehr E-Books als Hardcovers
		\item<4-> Q2/2011: ...mehr E-Books als gedruckte Bücher
	\end{itemize}
\end{frame}

\begin{frame}{Historische Geräte}
	\begin{columns}
		\column{.5\textwidth}
			\pgfimage[width=\textwidth]{bilder/Dynabook} \\
			Skizze Dynabook
		\column{.5\textwidth}
			\pgfimage[width=\textwidth]{bilder/flickr-81374383@N00-2262899712} \\
			Sony Librié (?), Rocketbook
	\end{columns}
\end{frame}

\begin{frame}{Aktuelle Reader}
	\begin{columns}
		\column{.5\textwidth}
			\pgfimage[height=.6\textheight]{bilder/flickr-40732538582@N01-3624716229} \\
			Sony PRS-Serie
		\column{.5\textwidth}
			\pgfimage[height=.6\textheight]{bilder/flickr-94168846@N00-4797684465} \\
			Kindle
	\end{columns}
\end{frame}

\begin{frame}{Der "`Heilbringer"'}
	\begin{columns}
		\column{.4\textwidth}
			\pgfimage[width=\textwidth]{bilder/flickr-16226024@N00-4508917013} \\
			Apple iPad
		\column{.6\textwidth}
			\begin{itemize}
				\item iPad wird bei der Ankündigung (auch) als \href{http://www.youtube.com/watch?v=gew68Qj5kxw}{E-Book-Lesegerät entdeckt}
				\item {\em "`Every publisher in the world should sit down once a day and pray to thank Steve Jobs that he is saving the publishing industry"'} -- M. Döpfner, CEO Axel Springer
				\item Durch Apples Medienpräsenz: Spätestens jetzt sind E-Books mainstream
			\end{itemize}
	\end{columns}
\end{frame}

\begin{frame}{Apples Politik}
	\begin{columns}
		\column{.3\textwidth}
			\pgfimage[width=\textwidth]{bilder/Steve_Jobs_with_the_Apple_iPad_no_logo} \\
		\column{.7\textwidth}
			\begin{itemize}
				\item E-Book-Reader mit Shopanbindung erscheinen fürs iPad. Bezahlung über Webseite der Betreiber
				\item Apple ändert AGBs: Extern zukaufbare Inhalte müssen zum selben Preis via App-Store verfügbar sein
				\item App-Store-Kauf: 30\% des Betrags geht an Apple
				\item Das ist bei Buchhändlern mehr als die Gewinnmarge
				\item Apple startet eigenen Bookstore
			\end{itemize}
	\end{columns}
\end{frame}

% vim: ai ts=2 sw=2



\section{Technik}

\begin{frame}{Technische Aspekte}
	...sind auch Entscheidungskriterien für den Kauf
	\begin{itemize}
		\item Gewicht
		\item Anzeigegröße (und Gerätegröße allgemein)
		\item Akku-Laufzeit
		\item Verarbeitungsgeschwindigkeit
		\item Netzanbindung (WLAN, UMTS)
		\item Speicher-Erweiterbarkeit
		\item Dateiformate (und deren Darstellungsqualität)
		\item Anbindung an einen Bücherladen
	\end{itemize}
	...aber vor allem:
\end{frame}

\subsection{Display-Technologien}

\begin{frame}{Displays!}
	\begin{columns}
		\column{.5\textwidth}
			Selbstleuchtend
			\smallskip

			\pgfimage[height=.6\textheight]{bilder/flickr-74028479@N00-282468560}
		\column{.5\textwidth}
			Reflektierend
			\smallskip

			\pgfimage[height=.6\textheight]{bilder/flickr-13631806@N00-1183612614}
	\end{columns}
\end{frame}

\begin{frame}{Selbstleuchtende Displays}
	\begin{columns}
		\column{.7\textwidth}
			Stark vereinfachter Aufbau:
			\begin{itemize}
				\item Hintergrundbeleuchtung
				\item Polarisationsfilter
				\item Flüssigkristallschicht
				\item Farbfilter
			\end{itemize}
			Beim iPad: "`Retina-Display"', IPS-Technik, Auflösung $>$300 dpi
		\column{.3\textwidth}
			\pgfimage[width=\textwidth]{bilder/LcdDeadPixel}
	\end{columns}
\end{frame}

\begin{frame}{Reflektive Displays}
	\begin{itemize}
		\item Mit LCD lassen sich auch reflektive (oder transreflektive) Displays bauen
		\item Wegen höherem Kontrast kommen e-Paper-Technologien zum Einsatz
		\item eInk ist übrigens eingetragenes Warenzeichen!
		\item Erste e-Papers: Drehbare Kügelchen (Gyricon)
		\item Heute bei fast allen e-Papers: Elektrophorese
	\end{itemize}
\end{frame}

\begin{frame}{Reflektive Displays}
	\begin{columns}
		\column{.6\textwidth}
			\begin{itemize}
				\item Reflektierender Hintergrund (8)
				\item Pixel-Elektroden-Schicht (7)
				\item Mikrokapseln (3)
				\item Schwarze und weiße Pigmente (4, 5), transparentes Öl (6)
				\item Transparente Elektrodenschicht (2)
				\item Oberflächenschutz, ggf. Farbfilter (1)
			\end{itemize}
		\column{.5\textwidth}
			\pgfimage[width=\textwidth]{bilder/Electronic_paper_Sideview}
	\end{columns}
\end{frame}

\begin{frame}{E-Ink (Kindle) unter dem Mikroskop}
	\begin{center}
		\pgfimage[width=.9\textwidth]{bilder/flickr-28594931@N03-5038428724}
	\end{center}
\end{frame}

\begin{frame}{Displaytechnologien im Vergleich}
	\begin{columns}
		\column{.45\textwidth}
			Selbstleuchtende Displays

			\begin{itemize}
				\item Schnelle Schaltzeiten (Animationen möglich)
				\item Farbe, hohe Farbbrillianz
				\item Hohe Auflösung
				\item Schlecht lesbar in heller Umgebung
				\item Ermüdet beim Lesen in dunkler Umgebung
			\end{itemize}

			\vfill
		\column{.55\textwidth}
			E-Paper-Displays

			\begin{itemize}
				\item Langsame Schaltzeiten (neue Controller verkürzen diese)
				\item Stromfluß nur bei Bildaufbau
				\item Leseeigenschaften wie Papier
				\item Geringere Auflösung (ca.\,200~dpi)
				\item Farbdisplays erst kurz vor Serienreife
			\end{itemize}

			\vfill
	\end{columns}
\end{frame}

% vim: ai ts=2 sw=2



\subsection{Dateiformate}

\begin{frame}{Futter für den E-Book-Reader: Dateiformate}
	Die verbreitetsten Formate:

	\begin{itemize}
		\item PDF (Portable Document Format)
		\item azw / topaz (Kindle-eigene Dateiformate)
		\item mobi (Mobipocket -- Basis für die Kindles azw)
		\item ePub (Electronic Publication)
	\end{itemize}

	...und viele weitere (Plucker, etc.).

	Sonderrolle: SiSu (Structured Information, Serialized Units): Meta-Format zum Generieren verschiedener Ausgabeformate
\end{frame}

\begin{frame}{Das "`ich-kann-alles"'-Format: PDF}
	\begin{itemize}
		\item Format seit 1993 von Adobe entwickelt
		\item Anfangs Nutzung gegen Lizenz, seit 2008 ISO-Standard
		\item Ursprünglicher Zweck: Geräteunabhängige Dokumentendarstellung, 100\%ige Layout-Treue
		\item Ständig wachsender Feature-Satz:
		\begin{itemize}
			\item Editierbarkeit, Kommentare
			\item Formulare
			\item Interaktivität mittels Javaskript
			\item Multimedia-Inhalte, 3d-CAD-Daten, Flash
			\item Reflow (Aufbrechen des Layouts)
			\item ...und vieles mehr
		\end{itemize}
	\end{itemize}
\end{frame}

\begin{frame}{ePub -- die "`Electronic PUBlication"'}
	\begin{itemize}
		\item Erste Spezifikation von 1999. v2.0 von 2007, aktuell: v2.0.1
		\item Relativ simples Format, auf den Zweck zugeschnitten
		\item Wiederverwendung vieler Standards und -technologien
		\begin{itemize}
			\item Konsequente Verwendung von Unicode
			\item Containerformat: zip
			\item Beschreibungsdaten: XML
			\item Inhalte: XHTML 1.1, CSS-2.0-Subset
			\item Grafikformate: png, jpeg, gif, svg
		\end{itemize}
	\end{itemize}
\end{frame}

% vim: ai ts=2 sw=2


% vim: ai ts=2 sw=2



\subsection{Displaygröße}

\begin{frame}{Display- und Gerätegröße}
	Display- und Gerätegröße hängt stark vom gewünschten Einsatzzweck ab.

	Typische Entscheidung zwischen:

	\begin{itemize}
		\item Groß, (oft) farbig, (meist) selbstleuchtend
		\item Klein, schwarz-weiß, (meist) E-Ink
	\end{itemize}
\end{frame}

\begin{frame}{Pro großes Display}
	\begin{itemize}
		\item Farbige Inhalte, u.U. multimediale Inhalte
		\item Bevorzugtes Lesematerial liegt (nur) als PDF vor
		\item Aufwendige Gestaltung des Inhalts oder 1:1-Umsetzungen von Printbüchern (bedeutet ebenfalls meist PDF)
		\item Typisch hierfür: Online-Versionen von Zeitschriften, Fachbücher
	\end{itemize}
\end{frame}

\begin{frame}{Pro kleines Display}
	\begin{itemize}
		\item Inhalte mit geringem Anteil an Grafiken und Schaubildern
		\item Lesematerial liegt in "`E-Reader-freundlichen"' Formaten (Text, html, epub, etc.) vor
		\item Automatische Umformatierung ideal für Fließtext
		\item Typisch hierfür: Romane, entsprechend aufbereitete Fachbücher
	\end{itemize}
\end{frame}

\begin{frame}{Was ist nun besser?}
	Was ist nun "`besser"'?
\begin{itemize}
		\item Es gibt kein Universalgerät
		\item Ähnlich Buchformat! Bücher werden ebenfalls unterschiedlich hergestellt...
		\begin{itemize}
			\item Größe
			\item Farbigkeit
			\item Papierqualität
		\end{itemize}
	\end{itemize}
\end{frame}


\section{DRM}

\begin{frame}{DRM und E-Books}
	\begin{itemize}
		\item Notwehr der Publizisten?
		\begin{itemize}
			\item Bindung von Geräten an einen Nutzer
			\item Lizensierung von Büchern an einen Nutzer
			\item Eigene Geräte können eigene Bücher ansehen
		\end{itemize}
		\item<2-> ...oder "`feuchter Traum"' der Publizisten?
		\begin{itemize}
			\item<2-> Geräte prüfen regelmäßig beim DRM-Server Legitimität des Zugriffs auf ein Buch
			\item<2-> Verleih von Büchern zwischen Nutzern
			\item<2-> Andere Geschäftsmodelle, z.B. Buchverleih
			\item<2-> "`Bestrafung"' von Lizenzübertretungen durch Deaktivieren des Zugriffs
			\item<2-> "`Rückruf"' von Inhalten (aus lizenzrechtlichen, politischen oder sonstigen Gründen)
		\end{itemize}
	\end{itemize}
\end{frame}

\begin{frame}{DRM und E-Books}
	\begin{block}{ }
		"`You cannot own an eBook. You can only license it."' \\
		--- Brit. Verleger
	\end{block}
\end{frame}


\newcommand{\cellyes}{\cellcolor{red}Ja}
\newcommand{\cellunknown}{\cellcolor{yellow}?}
\newcommand{\cellno}{\cellcolor{green}Nein}

\begin{frame}{Verschiedene DRM-Mechanismen}
	\begin{tabular}{lcccc}
	Verfolgbar/Trackbar & Kindle & Sony & iPad & Adobe \\
	Büchersuche & \cellunknown & \cellunknown & \cellyes & \cellno \\
	Buchkäufe & \cellyes & \cellno & \cellyes & \cellno \\
	Gelesenes Buch & \cellyes & \cellno & \cellno & \cellno \\
	Infos an Dritte & \cellyes & \cellyes & \cellyes & \cellno \\
	\end{tabular}
\end{frame}

\begin{frame}{Und was, wenn...}
	\begin{itemize}
		\item ...ich mein System neu aufsetzen muß? Backups! Geräteaktivierung!
		\item ...der DRM-Server nicht erreichbar ist oder abgeschaltet wurde?
		\item ...ich mein E-Book ausleihen möchte?
		\item ...ich mein E-Book verkaufen will?
		\item ...ich Bücher zu "`brisanten"' Themen lese?
		\item ...aufgrund von Repressionen ein Buch verschwinden soll?
	\end{itemize}
\end{frame}

\begin{frame}{DRM und E-Books}
	\begin{block}{Doctorow's First Law}
		"`Any time someone puts a lock on something that belongs to you, and won’t give you a key, they’re not doing it for your benefit."'
	\end{block}
\end{frame}

\begin{frame}{DRM mit Adobe Adept}
	\begin{itemize}
		\item Laut Prospekt: Bindung an User, Bindung an Gerät, Sicherung mit Passwort
		\item activation.xml (Oyo):
		\begin{itemize}
			\item Activation service: Service URL, Certificate, authentication certificate
			\item Credentials: User-ID, pkcs12-Key, license certificate, {\em private license key}, authentication certificate, username (Adobe ID)
			\item Activation token: Activation Service URL, Device-ID, User-ID, Signatur
			\item License service: License Service URL, Certificate
		\end{itemize}
		\item rights.xml (ePub):
		\begin{itemize}
			\item User-ID
			\item Resource-ID
			\item Device-ID
			\item Signature
			\item License Token
		\end{itemize}
	\end{itemize}
\end{frame}


\begin{frame}{DRM mit Adobe Adept}
	\begin{itemize}
		\item Gängige Praxis: Bücher mit User-Schlüssel verschlüsselt
		\item Symmetrischer Schlüssel: In Private License Key enthalten
		\item Info über verwendete Chiffre: In encryption.xml vermerkt
	\end{itemize}
\end{frame}

% vim: ai ts=2 sw=2



\section{E-Books und Bücherkultur}

\begin{frame}{Demokratisierung des Büchermarkts}
	Noch nie so einfach, selbst zu publizieren!
	\begin{itemize}
		\item Früher: Produktion im Voraus durch Verlag, hoher Aufwand
		\item<2-> Spezialisierte Verlage (z.B. für Promotionen)
		\item<3-> "`Print on demand"'-Dienste
		\item<4-> Verkauf von E-Books über verschiedene Plattformen
		\item<4-> ...oder kostenloser Download
	\end{itemize}
\end{frame}


\begin{frame}{Die Angst vor der Gratiskultur}
	\begin{itemize}
		\item Angst der Verlage vor dem digitalen Medium
		\item Insbesondere in Europa zögerliches Handeln
		\item Versuch, eigene Güter mit Kopierschutz zu schützen
		\item<2-> Auch reine Papierbücher werden gescannt und kopiert (Harry Potter!)
		\item<2-> Leute suchen längst nicht immer die billigste, sondern die bequemste Quelle
		\item<3-> Auch mit ungeschützten E-Books läßt sich Geld verdienen (Cory Doctorow, O'Reilly-Verlag, Harry Potter!)
	\end{itemize}
\end{frame}


\begin{frame}{Verlust an Ästhetik?}
	\begin{columns}
		\column{.5\textwidth}
			\pgfimage[height=.7\textheight]{bilder/Gutenberg_Bible_scan}
		\column{.5\textwidth}
			\pgfimage[height=.7\textheight]{bilder/flickr-34754790@N00-4738778284}
	\end{columns}
\end{frame}


\begin{frame}[plain]
	That's it!
	\begin{center}
		\pgfimage[height=6cm]{bilder/chaosknoten-fernsehturm}
	\end{center}
	\begin{center}
		Web: \url{http://stefan.ploing.de/} \\
		Twitter: \href{https://twitter.com/\_skyr}{@\_skyr}
	\end{center}
\end{frame}

\begin{frame}[plain]
  Image credits
  \tiny
  \begin{itemize}
    \item \url{http://blogs.smithsonianmag.com/paleofuture/2012/03/the-ipad-of-1935} Fair use
    \item \url{https://secure.wikimedia.org/wikipedia/en/wiki/File:Dynabook.png} Wikipedia Fair use
    \item \url{http://www.flickr.com/photos/cloudsoup/3624716229/} David Jones (CC) by
    \item \url{http://www.flickr.com/photos/librarycommission/2262899712/} Nebraska Library Commission (CC) by-nc-sa
    \item \url{http://www.flickr.com/photos/jimmiehomeschoolmom/4797684465/} Jimmie (CC) by
    \item \url{http://www.flickr.com/photos/fhke/4508917013/} FHKE (CC) by-sa
    \item \url{http://commons.wikimedia.org/wiki/File:Steve_Jobs_with_the_Apple_iPad_no_logo.jpg} Matt Buchanan (CC) by
    \item \url{http://www.flickr.com/photos/terretta/1183612614/} Michael Sean Terretta (CC) by-nc-nd
    \item \url{http://www.flickr.com/photos/flamsmark/282468560/} Flamsmark (CC) by-nc-sa
    \item \url{http://commons.wikimedia.org/wiki/File:Lcd_display_dead_pixel.jpg} Selçuk Oral (CC) by-sa
    \item \url{https://secure.wikimedia.org/wikipedia/en/wiki/File:Dynabook.png} Tosaka (CC) by
    \item \url{http://www.flickr.com/photos/28594931@N03/5038428724/} Specious Reasons (CC) by-nc
  \end{itemize}
\end{frame}


\end{document}
% vim: ai ts=2 sw=2

